% Options for packages loaded elsewhere
\PassOptionsToPackage{unicode}{hyperref}
\PassOptionsToPackage{hyphens}{url}
\PassOptionsToPackage{dvipsnames,svgnames*,x11names*}{xcolor}
%
\documentclass[
  english,
  man,floatsintext]{apa6}
\usepackage{lmodern}
\usepackage{amssymb,amsmath}
\usepackage{ifxetex,ifluatex}
\ifnum 0\ifxetex 1\fi\ifluatex 1\fi=0 % if pdftex
  \usepackage[T1]{fontenc}
  \usepackage[utf8]{inputenc}
  \usepackage{textcomp} % provide euro and other symbols
\else % if luatex or xetex
  \usepackage{unicode-math}
  \defaultfontfeatures{Scale=MatchLowercase}
  \defaultfontfeatures[\rmfamily]{Ligatures=TeX,Scale=1}
\fi
% Use upquote if available, for straight quotes in verbatim environments
\IfFileExists{upquote.sty}{\usepackage{upquote}}{}
\IfFileExists{microtype.sty}{% use microtype if available
  \usepackage[]{microtype}
  \UseMicrotypeSet[protrusion]{basicmath} % disable protrusion for tt fonts
}{}
\makeatletter
\@ifundefined{KOMAClassName}{% if non-KOMA class
  \IfFileExists{parskip.sty}{%
    \usepackage{parskip}
  }{% else
    \setlength{\parindent}{0pt}
    \setlength{\parskip}{6pt plus 2pt minus 1pt}}
}{% if KOMA class
  \KOMAoptions{parskip=half}}
\makeatother
\usepackage{xcolor}
\IfFileExists{xurl.sty}{\usepackage{xurl}}{} % add URL line breaks if available
\IfFileExists{bookmark.sty}{\usepackage{bookmark}}{\usepackage{hyperref}}
\hypersetup{
  pdftitle={Epidemiologische situatie COVID-19 in Nederland - 16 augustus 2021},
  pdfauthor={Marino van Zelst1},
  pdflang={en-EN},
  colorlinks=true,
  linkcolor=Maroon,
  filecolor=Maroon,
  citecolor=Blue,
  urlcolor=blue,
  pdfcreator={LaTeX via pandoc}}
\urlstyle{same} % disable monospaced font for URLs
\usepackage{graphicx,grffile}
\makeatletter
\def\maxwidth{\ifdim\Gin@nat@width>\linewidth\linewidth\else\Gin@nat@width\fi}
\def\maxheight{\ifdim\Gin@nat@height>\textheight\textheight\else\Gin@nat@height\fi}
\makeatother
% Scale images if necessary, so that they will not overflow the page
% margins by default, and it is still possible to overwrite the defaults
% using explicit options in \includegraphics[width, height, ...]{}
\setkeys{Gin}{width=\maxwidth,height=\maxheight,keepaspectratio}
% Set default figure placement to htbp
\makeatletter
\def\fps@figure{htbp}
\makeatother
\setlength{\emergencystretch}{3em} % prevent overfull lines
\providecommand{\tightlist}{%
  \setlength{\itemsep}{0pt}\setlength{\parskip}{0pt}}
\setcounter{secnumdepth}{-\maxdimen} % remove section numbering
% Make \paragraph and \subparagraph free-standing
\ifx\paragraph\undefined\else
  \let\oldparagraph\paragraph
  \renewcommand{\paragraph}[1]{\oldparagraph{#1}\mbox{}}
\fi
\ifx\subparagraph\undefined\else
  \let\oldsubparagraph\subparagraph
  \renewcommand{\subparagraph}[1]{\oldsubparagraph{#1}\mbox{}}
\fi
% Manuscript styling
\usepackage{upgreek}
\captionsetup{font=singlespacing,justification=justified}

% Table formatting
\usepackage{longtable}
\usepackage{lscape}
% \usepackage[counterclockwise]{rotating}   % Landscape page setup for large tables
\usepackage{multirow}		% Table styling
\usepackage{tabularx}		% Control Column width
\usepackage[flushleft]{threeparttable}	% Allows for three part tables with a specified notes section
\usepackage{threeparttablex}            % Lets threeparttable work with longtable

% Create new environments so endfloat can handle them
% \newenvironment{ltable}
%   {\begin{landscape}\begin{center}\begin{threeparttable}}
%   {\end{threeparttable}\end{center}\end{landscape}}
\newenvironment{lltable}{\begin{landscape}\begin{center}\begin{ThreePartTable}}{\end{ThreePartTable}\end{center}\end{landscape}}

% Enables adjusting longtable caption width to table width
% Solution found at http://golatex.de/longtable-mit-caption-so-breit-wie-die-tabelle-t15767.html
\makeatletter
\newcommand\LastLTentrywidth{1em}
\newlength\longtablewidth
\setlength{\longtablewidth}{1in}
\newcommand{\getlongtablewidth}{\begingroup \ifcsname LT@\roman{LT@tables}\endcsname \global\longtablewidth=0pt \renewcommand{\LT@entry}[2]{\global\advance\longtablewidth by ##2\relax\gdef\LastLTentrywidth{##2}}\@nameuse{LT@\roman{LT@tables}} \fi \endgroup}

% \setlength{\parindent}{0.5in}
% \setlength{\parskip}{0pt plus 0pt minus 0pt}

% \usepackage{etoolbox}
\makeatletter
\patchcmd{\HyOrg@maketitle}
  {\section{\normalfont\normalsize\abstractname}}
  {\section*{\normalfont\normalsize\abstractname}}
  {}{\typeout{Failed to patch abstract.}}
\patchcmd{\HyOrg@maketitle}
  {\section{\protect\normalfont{\@title}}}
  {\section*{\protect\normalfont{\@title}}}
  {}{\typeout{Failed to patch title.}}
\makeatother
\shorttitle{Dagelijkse rapportage}
\DeclareDelayedFloatFlavor{ThreePartTable}{table}
\DeclareDelayedFloatFlavor{lltable}{table}
\DeclareDelayedFloatFlavor*{longtable}{table}
\makeatletter
\renewcommand{\efloat@iwrite}[1]{\immediate\expandafter\protected@write\csname efloat@post#1\endcsname{}}
\makeatother
\usepackage{csquotes}
\ifxetex
  % Load polyglossia as late as possible: uses bidi with RTL langages (e.g. Hebrew, Arabic)
  \usepackage{polyglossia}
  \setmainlanguage[]{english}
\else
  \usepackage[shorthands=off,main=english]{babel}
\fi

\title{Epidemiologische situatie COVID-19 in Nederland - 16 augustus 2021}
\author{Marino van Zelst\textsuperscript{1}}
\date{}


\affiliation{\vspace{0.5cm}\textsuperscript{1} Vragen over deze rapportage kunnen verstuurd worden aan Marino van Zelst, twitter.com/mzelst. E-mail: \href{mailto:j.m.vanzelst@uvt.nl}{\nolinkurl{j.m.vanzelst@uvt.nl}}}

\begin{document}
\maketitle

{
\hypersetup{linkcolor=}
\setcounter{tocdepth}{3}
\tableofcontents
}
\newpage

\hypertarget{samenvatting}{%
\section{Samenvatting}\label{samenvatting}}

\textbf{Samenvatting (op basis van GGD cijfers)}\\
Tot en met 2021-08-16 zijn er in Nederland in totaal 1904128 COVID-19 patiënten gemeld aan het RIVM. Tot nu toe zijn 31744 van de gemelde patiënten opgenomen in het ziekenhuis en 17915 mensen overleden.

\textbf{Gegevens t. o. v. gisteren}\\
Positief getest: 2275
Totaal: 1904128 (+ 2228 ivm -47 corr.)

Opgenomen: 26
Totaal: 31744 (+
26 ivm 0 corr.)

Overleden: 6
Totaal: 17915 (+
6 ivm 0 corr.)

\textbf{Update met betrekking tot ziekenhuis-gegevens (data NICE)}

Patiënten verpleegafdeling\\
Bevestigd: 3 Verdacht: 3

Patiënten IC\\
Bevestigd: 5 Verdacht: 0

\textbf{Data}\\
Een databestand met de cumulatieve aantallen per gemeente per dag van gemelde COVID-19 patiënten, in het ziekenhuis opgenomen COVID-19 patiënten en overleden COVID-19 patiënten is \href{https://data.rivm.nl/geonetwork/srv/dut/catalog.search\#/metadata/1c0fcd57-1102-4620-9cfa-441e93ea5604}{hier} te vinden. Een databestand met karakteristieken van elke positief geteste COVID-19 patiënt in Nederland is \href{https://data.rivm.nl/geonetwork/srv/dut/catalog.search\#/metadata/2c4357c8-76e4-4662-9574-1deb8a73f724?tab=relations}{hier} te vinden. Alle gegevens die voor dit rapport gebruikt worden zijn te vinden in de \href{https://github.com/mzelst/covid-19}{Github repository}.

\newpage

\hypertarget{covid-19-meldingen-in-de-afgelopen-vier-weken}{%
\section{COVID-19 meldingen in de afgelopen vier weken}\label{covid-19-meldingen-in-de-afgelopen-vier-weken}}

\includegraphics{daily_report_files/figure-latex/Gemelde besmettingen-1.pdf}

\newpage

\hypertarget{kaart-met-covid-19-meldingen-per-gemeente-sinds-gisteren}{%
\section{Kaart met COVID-19 meldingen per gemeente sinds gisteren}\label{kaart-met-covid-19-meldingen-per-gemeente-sinds-gisteren}}

\begin{figure}
\centering
\includegraphics{daily_report_files/figure-latex/Gemeentes - sinds gisteren-1.pdf}
\caption{(\#Gemeentes - sinds gisteren) Aantal, sinds gisteren, bij de GGD'en gemelde COVID-19 patiënten per 100.000 inwoners per gemeente}
\end{figure}

\newpage

\hypertarget{kaart-met-covid-19-meldingen-per-gemeente-in-de-afgelopen-week}{%
\section{Kaart met COVID-19 meldingen per gemeente in de afgelopen week}\label{kaart-met-covid-19-meldingen-per-gemeente-in-de-afgelopen-week}}

\begin{figure}
\centering
\includegraphics{daily_report_files/figure-latex/Gemeentes - Sinds vorige week-1.pdf}
\caption{(\#Gemeentes - Sinds vorige week) Aantal in de afgelopen week bij de GGD'en gemelde COVID-19 patiënten per 100.000 inwoners per gemeente.}
\end{figure}

\newpage

\hypertarget{aantal-covid-19-meldingen-per-provincie-in-de-afgelopen-twee-weken}{%
\section{Aantal COVID-19 meldingen per provincie in de afgelopen twee weken}\label{aantal-covid-19-meldingen-per-provincie-in-de-afgelopen-twee-weken}}

\begin{table}
\centering
\begin{threeparttable}
\begin{tabular}{lrrrrrr}
\toprule
\multicolumn{1}{c}{ } & \multicolumn{2}{c}{Besmettingen} & \multicolumn{2}{c}{Ziekenhuisopnames} & \multicolumn{2}{c}{Overleden} \\
\cmidrule(l{3pt}r{3pt}){2-3} \cmidrule(l{3pt}r{3pt}){4-5} \cmidrule(l{3pt}r{3pt}){6-7}
Provincie & Totaal & /100000 & Totaal & /100000 & Totaal & /100000\\
\midrule
Drenthe & 583 & 117.6 & 3 & 0.6 & 0 & 0.0\\
Flevoland & 1151 & 267.2 & 12 & 2.8 & 5 & 1.2\\
Fryslân & 1057 & 162.2 & 11 & 1.7 & 3 & 0.5\\
Gelderland & 3432 & 163.4 & 27 & 1.3 & 21 & 1.0\\
Groningen & 895 & 152.5 & 5 & 0.9 & 1 & 0.2\\
Limburg & 2302 & 206.3 & 25 & 2.2 & 6 & 0.5\\
Noord-Brabant & 4395 & 170.5 & 26 & 1.0 & 4 & 0.2\\
Noord-Holland & 6944 & 240.4 & 85 & 2.9 & 16 & 0.6\\
Overijssel & 2248 & 192.5 & 9 & 0.8 & 3 & 0.3\\
Utrecht & 2295 & 168.5 & 49 & 3.6 & 6 & 0.4\\
Zeeland & 701 & 181.7 & 1 & 0.3 & 0 & 0.0\\
Zuid-Holland & 8216 & 220.2 & 92 & 2.5 & 21 & 0.6\\
\bottomrule
\end{tabular}
\begin{tablenotes}
\item \textit{Note: } 
\item Aantal bij de GGD’en gemelde COVID-19 patiënten, in het ziekenhuis opgenomen COVID-19 patiënten en overleden COVID-19 patiënten per provincie van 02 augustus t/m 16 augustus 10:00 uur, totaal en per 100.000 inwoners
\end{tablenotes}
\end{threeparttable}
\end{table}

\newpage

\hypertarget{aantal-covid-19-meldingen-per-ggd-in-de-afgelopen-twee-weken}{%
\section{Aantal COVID-19 meldingen per GGD in de afgelopen twee weken}\label{aantal-covid-19-meldingen-per-ggd-in-de-afgelopen-twee-weken}}

\begin{table}
\centering\begingroup\fontsize{10}{12}\selectfont

\begin{threeparttable}
\begin{tabular}{lrrrrrr}
\toprule
\multicolumn{1}{c}{ } & \multicolumn{2}{c}{Besmettingen} & \multicolumn{2}{c}{Ziekenhuisopnames} & \multicolumn{2}{c}{Overleden} \\
\cmidrule(l{3pt}r{3pt}){2-3} \cmidrule(l{3pt}r{3pt}){4-5} \cmidrule(l{3pt}r{3pt}){6-7}
GGD & Totaal & /100000 & Totaal & /100000 & Totaal & /100000\\
\midrule
Dienst Gezondheid \& Jeugd ZHZ & 911 & 197.3 & 4 & 0.9 & 1 & 0.2\\
GGD Amsterdam & 3493 & 326.3 & 49 & 4.6 & 8 & 0.7\\
GGD Brabant-Zuidoost & 1369 & 174.2 & 10 & 1.3 & 2 & 0.3\\
GGD Drenthe & 583 & 117.5 & 3 & 0.6 & 0 & 0.0\\
GGD Flevoland & 1156 & 268.2 & 12 & 2.8 & 5 & 1.2\\
GGD Fryslân & 1062 & 162.8 & 11 & 1.7 & 3 & 0.5\\
GGD Gelderland-Zuid & 896 & 158.5 & 10 & 1.8 & 5 & 0.9\\
GGD Gooi en Vechtstreek & 438 & 168.7 & 4 & 1.5 & 2 & 0.8\\
GGD Groningen & 847 & 144.3 & 3 & 0.5 & 1 & 0.2\\
GGD Haaglanden & 2542 & 226.0 & 44 & 3.9 & 12 & 1.1\\
GGD Hart voor Brabant & 1956 & 180.8 & 8 & 0.7 & 0 & 0.0\\
GGD Hollands-Midden & 1493 & 182.8 & 10 & 1.2 & 5 & 0.6\\
GGD Hollands-Noorden & 1151 & 172.4 & 8 & 1.2 & 1 & 0.1\\
GGD IJsselland & 893 & 166.6 & 4 & 0.7 & 0 & 0.0\\
GGD Kennemerland & 1113 & 201.6 & 13 & 2.4 & 3 & 0.5\\
GGD Limburg-Noord & 1024 & 196.0 & 9 & 1.7 & 3 & 0.6\\
GGD Noord- en Oost-Gelderland & 1345 & 161.5 & 8 & 1.0 & 8 & 1.0\\
GGD Regio Utrecht & 2303 & 364.5 & 48 & 7.6 & 6 & 0.9\\
GGD Rotterdam-Rijnmond & 3277 & 240.5 & 34 & 2.5 & 3 & 0.2\\
GGD Twente & 1356 & 102.0 & 5 & 0.4 & 3 & 0.2\\
GGD West-Brabant & 1077 & 151.3 & 8 & 1.1 & 2 & 0.3\\
GGD Zaanstreek/Waterland & 751 & 221.2 & 12 & 3.5 & 2 & 0.6\\
GGD Zeeland & 706 & 183.0 & 1 & 0.3 & 0 & 0.0\\
GGD Zuid-Limburg & 1284 & 216.5 & 18 & 3.0 & 3 & 0.5\\
Veiligheids- en Gezondheidsregio Gelderland-Midden & 1193 & 169.7 & 9 & 1.3 & 8 & 1.1\\
\bottomrule
\end{tabular}
\begin{tablenotes}
\item \textit{Note: } 
\item Aantal bij de GGD’en gemelde COVID-19 patiënten, in het ziekenhuis opgenomen COVID-19 patiënten en overleden COVID-19 patiënten per GGD van 02 augustus t/m 16 augustus 10:00 uur, totaal en per 100.000 inwoners
\end{tablenotes}
\end{threeparttable}
\endgroup{}
\end{table}

\newpage

\hypertarget{ziekenhuisopnames-nice-in-de-afgelopen-twee-weken}{%
\section{Ziekenhuisopnames (NICE) in de afgelopen twee weken}\label{ziekenhuisopnames-nice-in-de-afgelopen-twee-weken}}

\begin{table}
\centering\begingroup\fontsize{10}{12}\selectfont

\begin{threeparttable}
\begin{tabular}{lrrrrrr}
\toprule
\multicolumn{1}{c}{ } & \multicolumn{2}{c}{Aanwezig} & \multicolumn{2}{c}{Opnames} & \multicolumn{2}{c}{Overleden} \\
\cmidrule(l{3pt}r{3pt}){2-3} \cmidrule(l{3pt}r{3pt}){4-5} \cmidrule(l{3pt}r{3pt}){6-7}
Leeftijd & Kliniek & IC & Kliniek & IC & Kliniek & IC\\
\midrule
<20 & 29 & 0 & 48 & 1 & 0 & 0\\
20 - 24 & 9 & 2 & 20 & 5 & 0 & 0\\
25 - 29 & 20 & 6 & 38 & 13 & 0 & 0\\
30 - 34 & 12 & 4 & 32 & 8 & 0 & 0\\
35 - 39 & 19 & 10 & 35 & 11 & 0 & 0\\
40 - 44 & 23 & 7 & 48 & 12 & 0 & 0\\
45 - 49 & 16 & 13 & 40 & 16 & 1 & 2\\
50 - 54 & 36 & 26 & 64 & 22 & 1 & 3\\
55 - 59 & 39 & 21 & 61 & 18 & 0 & 2\\
60 - 64 & 47 & 26 & 77 & 27 & 2 & 6\\
65 - 69 & 42 & 40 & 47 & 24 & 7 & 5\\
70 - 74 & 38 & 29 & 30 & 27 & 5 & 7\\
75 - 79 & 47 & 18 & 57 & 15 & 11 & 3\\
80 - 84 & 55 & 1 & 59 & 1 & 15 & 0\\
85 - 89 & 35 & 0 & 51 & 0 & 12 & 1\\
>90 & 14 & 0 & 21 & 0 & 8 & 0\\
\bottomrule
\end{tabular}
\begin{tablenotes}
\item \textit{Note: } 
\item Aantal bij NICE gemelde COVID-19 patiënten: in het ziekenhuis aanwezige COVID-19 patiënten, opnames, en overleden COVID-19 patiënten in het ziekenhuis van 02 augustus t/m 16 augustus 15:15 uur
\end{tablenotes}
\end{threeparttable}
\endgroup{}
\end{table}

\newpage

\hypertarget{leeftijdsverdeling-en-man-vrouwverdeling-van-covid-19-patiuxebnten-in-de-afgelopen-twee-weken}{%
\section{Leeftijdsverdeling en man-vrouwverdeling van COVID-19 patiënten in de afgelopen twee weken}\label{leeftijdsverdeling-en-man-vrouwverdeling-van-covid-19-patiuxebnten-in-de-afgelopen-twee-weken}}

\begin{table}
\centering\begingroup\fontsize{11}{13}\selectfont

\begin{threeparttable}
\begin{tabular}{lrrrrrr}
\toprule
\multicolumn{1}{c}{ } & \multicolumn{2}{c}{Besmettingen} & \multicolumn{2}{c}{Ziekenhuisopnames} & \multicolumn{2}{c}{Overleden} \\
\cmidrule(l{3pt}r{3pt}){2-3} \cmidrule(l{3pt}r{3pt}){4-5} \cmidrule(l{3pt}r{3pt}){6-7}
Leeftijdsgroep & Totaal & \% & Totaal & \% & Totaal & \%\\
\midrule
<50 & 3 & 0.0 & 0 & 0.0 & 3 & 3.5\\
0-9 & 1816 & 5.3 & 14 & 4.1 & 0 & 0.0\\
10-19 & 6871 & 20.1 & 7 & 2.0 & 0 & 0.0\\
20-29 & 9633 & 28.2 & 24 & 7.0 & 0 & 0.0\\
30-39 & 5522 & 16.1 & 24 & 7.0 & 0 & 0.0\\
40-49 & 3729 & 10.9 & 40 & 11.6 & 0 & 0.0\\
50-59 & 3192 & 9.3 & 53 & 15.4 & 6 & 7.0\\
60-69 & 1895 & 5.5 & 70 & 20.3 & 7 & 8.1\\
70-79 & 861 & 2.5 & 57 & 16.5 & 11 & 12.8\\
80-89 & 533 & 1.6 & 46 & 13.3 & 37 & 43.0\\
90+ & 164 & 0.5 & 10 & 2.9 & 22 & 25.6\\
\bottomrule
\end{tabular}
\begin{tablenotes}
\item[1] Leeftijdsverdeling van bij de GGD’en gemelde COVID-19 patiënten, in het ziekenhuis opgenomen COVID-19 patiënten en overleden COVID-19 patiënten van 02 augustus t/m 16 augustus 10:00 uur.
\item[2] Overlijdensgevallen van patiënten jonger dan 50 jaar oud worden door het RIVM in de casusdata gegroepeerd in de categorie <50. Deze overlijdensgevallen zijn hier dus niet zichtbaar in deze leeftijdsgroepen maar alleen via de groep '<50'.
\end{tablenotes}
\end{threeparttable}
\endgroup{}
\end{table}

\newpage

\begin{table}
\centering\begingroup\fontsize{11}{13}\selectfont

\begin{threeparttable}
\begin{tabular}{lrrrrrr}
\toprule
\multicolumn{1}{c}{ } & \multicolumn{2}{c}{Besmettingen} & \multicolumn{2}{c}{Ziekenhuisopnames} & \multicolumn{2}{c}{Overleden} \\
\cmidrule(l{3pt}r{3pt}){2-3} \cmidrule(l{3pt}r{3pt}){4-5} \cmidrule(l{3pt}r{3pt}){6-7}
Geslacht & Totaal & \% & Totaal & \% & Totaal & \%\\
\midrule
Vrouw & 17303 & 50.6 & 156 & 45.2 & 39 & 45.3\\
Man & 16916 & 49.4 & 189 & 54.8 & 47 & 54.7\\
Onbekend & 0 & 0.0 & 0 & 0.0 & 0 & 0.0\\
\bottomrule
\end{tabular}
\begin{tablenotes}
\item \textit{Note: } 
\item Man-vrouwverdeling van bij de GGD’en gemelde COVID-19 patiënten, in het ziekenhuis opgenomen COVID-19 patiënten en overleden COVID-19 patiënten van 02 augustus t/m 16 augustus 10:00 uur.
\end{tablenotes}
\end{threeparttable}
\endgroup{}
\end{table}
\newpage


\end{document}
