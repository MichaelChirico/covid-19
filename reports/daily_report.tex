% Options for packages loaded elsewhere
\PassOptionsToPackage{unicode}{hyperref}
\PassOptionsToPackage{hyphens}{url}
\PassOptionsToPackage{dvipsnames,svgnames*,x11names*}{xcolor}
%
\documentclass[
  english,
  man,floatsintext]{apa6}
\usepackage{lmodern}
\usepackage{amssymb,amsmath}
\usepackage{ifxetex,ifluatex}
\ifnum 0\ifxetex 1\fi\ifluatex 1\fi=0 % if pdftex
  \usepackage[T1]{fontenc}
  \usepackage[utf8]{inputenc}
  \usepackage{textcomp} % provide euro and other symbols
\else % if luatex or xetex
  \usepackage{unicode-math}
  \defaultfontfeatures{Scale=MatchLowercase}
  \defaultfontfeatures[\rmfamily]{Ligatures=TeX,Scale=1}
\fi
% Use upquote if available, for straight quotes in verbatim environments
\IfFileExists{upquote.sty}{\usepackage{upquote}}{}
\IfFileExists{microtype.sty}{% use microtype if available
  \usepackage[]{microtype}
  \UseMicrotypeSet[protrusion]{basicmath} % disable protrusion for tt fonts
}{}
\makeatletter
\@ifundefined{KOMAClassName}{% if non-KOMA class
  \IfFileExists{parskip.sty}{%
    \usepackage{parskip}
  }{% else
    \setlength{\parindent}{0pt}
    \setlength{\parskip}{6pt plus 2pt minus 1pt}}
}{% if KOMA class
  \KOMAoptions{parskip=half}}
\makeatother
\usepackage{xcolor}
\IfFileExists{xurl.sty}{\usepackage{xurl}}{} % add URL line breaks if available
\IfFileExists{bookmark.sty}{\usepackage{bookmark}}{\usepackage{hyperref}}
\hypersetup{
  pdftitle={Epidemiologische situatie COVID-19 in Nederland - 20 maart 2022},
  pdfauthor={Marino van Zelst1},
  pdflang={en-EN},
  colorlinks=true,
  linkcolor=Maroon,
  filecolor=Maroon,
  citecolor=Blue,
  urlcolor=blue,
  pdfcreator={LaTeX via pandoc}}
\urlstyle{same} % disable monospaced font for URLs
\usepackage{graphicx,grffile}
\makeatletter
\def\maxwidth{\ifdim\Gin@nat@width>\linewidth\linewidth\else\Gin@nat@width\fi}
\def\maxheight{\ifdim\Gin@nat@height>\textheight\textheight\else\Gin@nat@height\fi}
\makeatother
% Scale images if necessary, so that they will not overflow the page
% margins by default, and it is still possible to overwrite the defaults
% using explicit options in \includegraphics[width, height, ...]{}
\setkeys{Gin}{width=\maxwidth,height=\maxheight,keepaspectratio}
% Set default figure placement to htbp
\makeatletter
\def\fps@figure{htbp}
\makeatother
\setlength{\emergencystretch}{3em} % prevent overfull lines
\providecommand{\tightlist}{%
  \setlength{\itemsep}{0pt}\setlength{\parskip}{0pt}}
\setcounter{secnumdepth}{-\maxdimen} % remove section numbering
% Make \paragraph and \subparagraph free-standing
\ifx\paragraph\undefined\else
  \let\oldparagraph\paragraph
  \renewcommand{\paragraph}[1]{\oldparagraph{#1}\mbox{}}
\fi
\ifx\subparagraph\undefined\else
  \let\oldsubparagraph\subparagraph
  \renewcommand{\subparagraph}[1]{\oldsubparagraph{#1}\mbox{}}
\fi
% Manuscript styling
\usepackage{upgreek}
\captionsetup{font=singlespacing,justification=justified}

% Table formatting
\usepackage{longtable}
\usepackage{lscape}
% \usepackage[counterclockwise]{rotating}   % Landscape page setup for large tables
\usepackage{multirow}		% Table styling
\usepackage{tabularx}		% Control Column width
\usepackage[flushleft]{threeparttable}	% Allows for three part tables with a specified notes section
\usepackage{threeparttablex}            % Lets threeparttable work with longtable

% Create new environments so endfloat can handle them
% \newenvironment{ltable}
%   {\begin{landscape}\begin{center}\begin{threeparttable}}
%   {\end{threeparttable}\end{center}\end{landscape}}
\newenvironment{lltable}{\begin{landscape}\begin{center}\begin{ThreePartTable}}{\end{ThreePartTable}\end{center}\end{landscape}}

% Enables adjusting longtable caption width to table width
% Solution found at http://golatex.de/longtable-mit-caption-so-breit-wie-die-tabelle-t15767.html
\makeatletter
\newcommand\LastLTentrywidth{1em}
\newlength\longtablewidth
\setlength{\longtablewidth}{1in}
\newcommand{\getlongtablewidth}{\begingroup \ifcsname LT@\roman{LT@tables}\endcsname \global\longtablewidth=0pt \renewcommand{\LT@entry}[2]{\global\advance\longtablewidth by ##2\relax\gdef\LastLTentrywidth{##2}}\@nameuse{LT@\roman{LT@tables}} \fi \endgroup}

% \setlength{\parindent}{0.5in}
% \setlength{\parskip}{0pt plus 0pt minus 0pt}

% \usepackage{etoolbox}
\makeatletter
\patchcmd{\HyOrg@maketitle}
  {\section{\normalfont\normalsize\abstractname}}
  {\section*{\normalfont\normalsize\abstractname}}
  {}{\typeout{Failed to patch abstract.}}
\patchcmd{\HyOrg@maketitle}
  {\section{\protect\normalfont{\@title}}}
  {\section*{\protect\normalfont{\@title}}}
  {}{\typeout{Failed to patch title.}}
\makeatother
\shorttitle{Dagelijkse rapportage}
\DeclareDelayedFloatFlavor{ThreePartTable}{table}
\DeclareDelayedFloatFlavor{lltable}{table}
\DeclareDelayedFloatFlavor*{longtable}{table}
\makeatletter
\renewcommand{\efloat@iwrite}[1]{\immediate\expandafter\protected@write\csname efloat@post#1\endcsname{}}
\makeatother
\usepackage{csquotes}
\ifxetex
  % Load polyglossia as late as possible: uses bidi with RTL langages (e.g. Hebrew, Arabic)
  \usepackage{polyglossia}
  \setmainlanguage[]{english}
\else
  \usepackage[shorthands=off,main=english]{babel}
\fi

\title{Epidemiologische situatie COVID-19 in Nederland - 20 maart 2022}
\author{Marino van Zelst\textsuperscript{1}}
\date{}


\affiliation{\vspace{0.5cm}\textsuperscript{1} Vragen over deze rapportage kunnen verstuurd worden aan Marino van Zelst, twitter.com/mzelst. E-mail: \href{mailto:marino.vanzelst@wur.nl}{\nolinkurl{marino.vanzelst@wur.nl}}}

\begin{document}
\maketitle

{
\hypersetup{linkcolor=}
\setcounter{tocdepth}{3}
\tableofcontents
}
\newpage

\hypertarget{samenvatting}{%
\section{Samenvatting}\label{samenvatting}}

\textbf{Samenvatting (op basis van GGD cijfers)}\\
Tot en met 2022-03-20 zijn er in Nederland in totaal 7511397 COVID-19 patiënten gemeld aan het RIVM. Tot nu toe zijn er 21793 mensen overleden.

\textbf{Gegevens t. o. v. gisteren}\\
Positief getest: 34846
Totaal: 7511397 (+ 34823 ivm -23 corr.)

Overleden: 8
Totaal: 21793 (+
8 ivm 0 corr.)

\textbf{Update met betrekking tot ziekenhuis-gegevens (data NICE)}

Patiënten verpleegafdeling\\
Bevestigd: 4 Verdacht: 0

Patiënten IC\\
Bevestigd: 0 Verdacht: 0

\textbf{Data}\\
Een databestand met de cumulatieve aantallen per gemeente per dag van gemelde COVID-19 patiënten, in het ziekenhuis opgenomen COVID-19 patiënten en overleden COVID-19 patiënten is \href{https://data.rivm.nl/geonetwork/srv/dut/catalog.search\#/metadata/1c0fcd57-1102-4620-9cfa-441e93ea5604}{hier} te vinden. Een databestand met karakteristieken van elke positief geteste COVID-19 patiënt in Nederland is \href{https://data.rivm.nl/geonetwork/srv/dut/catalog.search\#/metadata/2c4357c8-76e4-4662-9574-1deb8a73f724?tab=relations}{hier} te vinden. Alle gegevens die voor dit rapport gebruikt worden zijn te vinden in de \href{https://github.com/mzelst/covid-19}{Github repository}.

\newpage

\hypertarget{covid-19-meldingen-in-de-afgelopen-vier-weken}{%
\section{COVID-19 meldingen in de afgelopen vier weken}\label{covid-19-meldingen-in-de-afgelopen-vier-weken}}

\includegraphics{daily_report_files/figure-latex/Gemelde besmettingen-1.pdf}

\newpage

\hypertarget{kaart-met-covid-19-meldingen-per-gemeente-sinds-gisteren}{%
\section{Kaart met COVID-19 meldingen per gemeente sinds gisteren}\label{kaart-met-covid-19-meldingen-per-gemeente-sinds-gisteren}}

\begin{figure}
\centering
\includegraphics{daily_report_files/figure-latex/Gemeentes - sinds gisteren-1.pdf}
\caption{(\#Gemeentes - sinds gisteren) Aantal, sinds gisteren, bij de GGD'en gemelde COVID-19 patiënten per 100.000 inwoners per gemeente}
\end{figure}

\newpage

\hypertarget{kaart-met-covid-19-meldingen-per-gemeente-in-de-afgelopen-week}{%
\section{Kaart met COVID-19 meldingen per gemeente in de afgelopen week}\label{kaart-met-covid-19-meldingen-per-gemeente-in-de-afgelopen-week}}

\begin{figure}
\centering
\includegraphics{daily_report_files/figure-latex/Gemeentes - Sinds vorige week-1.pdf}
\caption{(\#Gemeentes - Sinds vorige week) Aantal in de afgelopen week bij de GGD'en gemelde COVID-19 patiënten per 100.000 inwoners per gemeente.}
\end{figure}

\newpage

\hypertarget{aantal-covid-19-meldingen-per-provincie-in-de-afgelopen-twee-weken}{%
\section{Aantal COVID-19 meldingen per provincie in de afgelopen twee weken}\label{aantal-covid-19-meldingen-per-provincie-in-de-afgelopen-twee-weken}}

\begin{table}
\centering
\begin{threeparttable}
\begin{tabular}{lrrrr}
\toprule
\multicolumn{1}{c}{ } & \multicolumn{2}{c}{Besmettingen} & \multicolumn{2}{c}{Overleden} \\
\cmidrule(l{3pt}r{3pt}){2-3} \cmidrule(l{3pt}r{3pt}){4-5}
Provincie & Totaal & /100000 & Totaal & /100000\\
\midrule
Drenthe & 27485 & 5522.4 & 4 & 0.8\\
Flevoland & 14792 & 3407.9 & 1 & 0.2\\
Fryslân & 30566 & 4671.5 & 13 & 2.0\\
Gelderland & 111815 & 5299.9 & 29 & 1.4\\
Groningen & 31094 & 5263.6 & 4 & 0.7\\
Limburg & 70358 & 6285.2 & 17 & 1.5\\
Noord-Brabant & 137702 & 5312.9 & 20 & 0.8\\
Noord-Holland & 111015 & 3819.1 & 27 & 0.9\\
Overijssel & 61671 & 5264.4 & 9 & 0.8\\
Utrecht & 60637 & 4429.6 & 3 & 0.2\\
Zeeland & 16920 & 4373.5 & 6 & 1.6\\
Zuid-Holland & 130139 & 3469.8 & 48 & 1.3\\
\bottomrule
\end{tabular}
\begin{tablenotes}
\item \textit{Note: } 
\item Aantal bij de GGD’en gemelde COVID-19 patiënten en overleden COVID-19 patiënten per provincie van 06 maart t/m 20 maart 10:00 uur, totaal en per 100.000 inwoners
\end{tablenotes}
\end{threeparttable}
\end{table}

\newpage

\hypertarget{aantal-covid-19-meldingen-per-ggd-in-de-afgelopen-twee-weken}{%
\section{Aantal COVID-19 meldingen per GGD in de afgelopen twee weken}\label{aantal-covid-19-meldingen-per-ggd-in-de-afgelopen-twee-weken}}

\begin{table}
\centering\begingroup\fontsize{10}{12}\selectfont

\begin{threeparttable}
\begin{tabular}{lrrrr}
\toprule
\multicolumn{1}{c}{ } & \multicolumn{2}{c}{Besmettingen} & \multicolumn{2}{c}{Overleden} \\
\cmidrule(l{3pt}r{3pt}){2-3} \cmidrule(l{3pt}r{3pt}){4-5}
GGD & Totaal & /100000 & Totaal & /100000\\
\midrule
Dienst Gezondheid \& Jeugd ZHZ & 15273 & 3295.4 & 2 & 0.4\\
GGD Amsterdam & 41209 & 3796.9 & 14 & 1.3\\
GGD Brabant-Zuidoost & 42069 & 5318.2 & 4 & 0.5\\
GGD Drenthe & 27455 & 5515.3 & 4 & 0.8\\
GGD Flevoland & 14784 & 3402.7 & 1 & 0.2\\
GGD Fryslân & 30589 & 4675.8 & 13 & 2.0\\
GGD Gelderland-Zuid & 30707 & 5396.5 & 9 & 1.6\\
GGD Gooi en Vechtstreek & 9626 & 3692.2 & 2 & 0.8\\
GGD Groningen & 31033 & 5255.0 & 4 & 0.7\\
GGD Haaglanden & 39050 & 3447.7 & 3 & 0.3\\
GGD Hart voor Brabant & 59990 & 5518.0 & 11 & 1.0\\
GGD Hollands-Midden & 32866 & 4005.3 & 5 & 0.6\\
GGD Hollands-Noorden & 24767 & 3697.7 & 1 & 0.1\\
GGD IJsselland & 30431 & 5658.3 & 8 & 1.5\\
GGD Kennemerland & 23069 & 4164.5 & 4 & 0.7\\
GGD Limburg-Noord & 34447 & 6570.3 & 4 & 0.8\\
GGD Noord- en Oost-Gelderland & 44810 & 5367.1 & 15 & 1.8\\
GGD Regio Utrecht & 60678 & 4429.7 & 3 & 0.2\\
GGD Rotterdam-Rijnmond & 43027 & 3220.0 & 39 & 2.9\\
GGD Twente & 31303 & 4938.1 & 1 & 0.2\\
GGD West-Brabant & 34474 & 4824.4 & 6 & 0.8\\
GGD Zaanstreek/Waterland & 12350 & 3624.8 & 4 & 1.2\\
GGD Zeeland & 16929 & 4376.1 & 6 & 1.6\\
GGD Zuid-Limburg & 36499 & 6135.7 & 13 & 2.2\\
Veiligheids- en Gezondheidsregio Gelderland-Midden & 36755 & 5202.0 & 5 & 0.7\\
\bottomrule
\end{tabular}
\begin{tablenotes}
\item \textit{Note: } 
\item Aantal bij de GGD’en gemelde COVID-19 patiënten en overleden COVID-19 patiënten per GGD van 06 maart t/m 20 maart 10:00 uur, totaal en per 100.000 inwoners
\end{tablenotes}
\end{threeparttable}
\endgroup{}
\end{table}

\newpage

\hypertarget{ziekenhuisopnames-nice-in-de-afgelopen-twee-weken}{%
\section{Ziekenhuisopnames (NICE) in de afgelopen twee weken}\label{ziekenhuisopnames-nice-in-de-afgelopen-twee-weken}}

\begin{table}
\centering\begingroup\fontsize{10}{12}\selectfont

\begin{threeparttable}
\begin{tabular}{lrrrrrr}
\toprule
\multicolumn{1}{c}{ } & \multicolumn{2}{c}{Aanwezig} & \multicolumn{2}{c}{Opnames} & \multicolumn{2}{c}{Overleden} \\
\cmidrule(l{3pt}r{3pt}){2-3} \cmidrule(l{3pt}r{3pt}){4-5} \cmidrule(l{3pt}r{3pt}){6-7}
Leeftijd & Kliniek & IC & Kliniek & IC & Kliniek & IC\\
\midrule
<20 & 87 & 2 & 275 & 4 & 0 & 0\\
20 - 24 & 11 & 2 & 47 & 6 & 0 & 0\\
25 - 29 & 29 & 1 & 106 & 1 & 0 & 0\\
30 - 34 & 42 & 3 & 124 & 6 & 0 & 0\\
35 - 39 & 30 & 2 & 96 & 1 & 0 & 0\\
40 - 44 & 30 & 6 & 70 & 6 & 0 & 1\\
45 - 49 & 39 & 9 & 89 & 10 & 3 & 0\\
50 - 54 & 62 & 9 & 132 & 9 & 4 & 4\\
55 - 59 & 103 & 13 & 193 & 14 & 3 & 7\\
60 - 64 & 139 & 22 & 249 & 24 & 8 & 14\\
65 - 69 & 195 & 25 & 322 & 44 & 13 & 14\\
70 - 74 & 229 & 32 & 401 & 43 & 33 & 15\\
75 - 79 & 247 & 18 & 414 & 26 & 37 & 10\\
80 - 84 & 263 & 5 & 478 & 15 & 49 & 4\\
85 - 89 & 200 & 0 & 366 & 2 & 56 & 1\\
>90 & 102 & 0 & 195 & 0 & 31 & 0\\
\bottomrule
\end{tabular}
\begin{tablenotes}
\item \textit{Note: } 
\item Aantal bij NICE gemelde COVID-19 patiënten: in het ziekenhuis aanwezige COVID-19 patiënten, opnames, en overleden COVID-19 patiënten in het ziekenhuis van 06 maart t/m 20 maart 15:15 uur
\end{tablenotes}
\end{threeparttable}
\endgroup{}
\end{table}

\newpage

\hypertarget{leeftijdsverdeling-en-man-vrouwverdeling-van-covid-19-patiuxebnten-in-de-afgelopen-twee-weken}{%
\section{Leeftijdsverdeling en man-vrouwverdeling van COVID-19 patiënten in de afgelopen twee weken}\label{leeftijdsverdeling-en-man-vrouwverdeling-van-covid-19-patiuxebnten-in-de-afgelopen-twee-weken}}

\begin{table}
\centering\begingroup\fontsize{11}{13}\selectfont

\begin{threeparttable}
\begin{tabular}{lrrrr}
\toprule
\multicolumn{1}{c}{ } & \multicolumn{2}{c}{Besmettingen} & \multicolumn{2}{c}{Overleden} \\
\cmidrule(l{3pt}r{3pt}){2-3} \cmidrule(l{3pt}r{3pt}){4-5}
Leeftijdsgroep & Totaal & \% & Totaal & \%\\
\midrule
<50 & 4 & 0.0 & 4 & 2.2\\
0-9 & 29882 & 3.7 & NA & NA\\
10-19 & 76786 & 9.5 & NA & NA\\
20-29 & 139221 & 17.3 & NA & NA\\
30-39 & 140084 & 17.4 & NA & NA\\
40-49 & 116274 & 14.5 & NA & NA\\
50-59 & 134241 & 16.7 & 8 & 4.4\\
60-69 & 90975 & 11.3 & 18 & 9.9\\
70-79 & 53497 & 6.7 & 42 & 23.2\\
80-89 & 18878 & 2.3 & 70 & 38.7\\
90+ & 4327 & 0.5 & 39 & 21.5\\
\bottomrule
\end{tabular}
\begin{tablenotes}
\item[1] Leeftijdsverdeling van bij de GGD’en gemelde COVID-19 patiënten en overleden COVID-19 patiënten van 06 maart t/m 20 maart 10:00 uur.
\item[2] Overlijdensgevallen van patiënten jonger dan 50 jaar oud worden door het RIVM in de casusdata gegroepeerd in de categorie <50. Deze overlijdensgevallen zijn hier dus niet zichtbaar in deze leeftijdsgroepen maar alleen via de groep '<50'.
\end{tablenotes}
\end{threeparttable}
\endgroup{}
\end{table}

\newpage

\begin{table}
\centering\begingroup\fontsize{11}{13}\selectfont

\begin{threeparttable}
\begin{tabular}{lrrrr}
\toprule
\multicolumn{1}{c}{ } & \multicolumn{2}{c}{Besmettingen} & \multicolumn{2}{c}{Overleden} \\
\cmidrule(l{3pt}r{3pt}){2-3} \cmidrule(l{3pt}r{3pt}){4-5}
Geslacht & Totaal & \% & Totaal & \%\\
\midrule
Vrouw & 428599 & 53.3 & 79 & 43.6\\
Man & 374422 & 46.6 & 102 & 56.4\\
Onbekend & 1174 & 0.1 & NA & NA\\
\bottomrule
\end{tabular}
\begin{tablenotes}
\item \textit{Note: } 
\item Man-vrouwverdeling van bij de GGD’en gemelde COVID-19 patiënten en overleden COVID-19 patiënten van 06 maart t/m 20 maart 10:00 uur.
\end{tablenotes}
\end{threeparttable}
\endgroup{}
\end{table}
\newpage


\end{document}
